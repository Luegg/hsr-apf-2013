\subsection{Controlled Process Creator}
\label{sec:controlled-process-creator}

Der \emph{Controlled Process Creator} kümmert sich um die Erstellung neuer Prozess in einem Betriebssystem und stellt sicher, dass diesen korrekte Zugriffsberechtigungen zugewiesen werden.

\subsection*{Kontext}
Ein Betriebssystem, welches Prozesse und Threads für Anwendungen erstellen kann.

\subsection*{Problem}
Erstellt ein Betriebssystem einen neuen Prozess oder Thread, so erbt dieser im einfachsten Falle die Berechtigungen des übergeordneten Prozesses/Thread.

Dies kann sehr schnell zum Sicherheitsrisiko werden, kann sich ein Angreifer so doch ziemlich schnell Zugriff aufs gesamte System verschaffen.

Wie kann sichergestellt werden, dass jedem neuen Prozess oder Thread entsprechend aufgestellten Sicherheitsrichtlinien Berechtigungen zugeteilt werden? Ferner sind folgende Punkte zu beachten:

\begin{itemize}
	\item Die Auswahl von Berechtigungen für neue Prozesse soll so einfach wie möglich sein
	\item Die Definition der Richtlinien, welche Prozesse welche Berechtigungen erhalten sollen, soll flexibel und einfach sein.
	\item Ein Prozess, welcher einem anderen untergeordnet ist, soll die Möglichkeit haben, die Berechtigungen des Übergeordneten zu übernehmen. Dies jedoch kontrolliert und unter Voraussetzung entsprechender Sicherheitsmassnahmen.
	\item Die Anzahl untergeordneter Prozesse soll beschränkt werden, um Denail of Service Attacken zu verhindern.
	\item Ausnahmen müssen handlebar sein: Will ein Prozess auf etwas zugreifen, worauf er keinen Zugriff hat, soll dies irgendwie ermöglicht werden.
\end{itemize}


\subsection*{Lösung}
Prozesse werden im Normalfall direkt vom Betriebssystem selber über entsprechende Funktionen erstellt. Der \emph{Controlled Process Creator} setzt hier an und stellt sicher, dass jedem neuen Prozess korrekte Berechtigungen zugewiesen werden.

Weiter kann der übergeordnete Prozess seinem neuen untergeordneten Prozess ein Subset seiner eigenen Berechtigungen zuweisen.


\subsection*{Vorteile}
\begin{itemize}
	\item Ein neuer Prozess erhält nur die Berechtigungen, welche den definierten Richtlinien entsprechen
	\item Die Anzahl möglicher untergeordneten Prozesse kann zentral beschränkt werden
	\item Berechtigungen können den übergeordneten Prozess resp. dessen ID enthalten. So können, falls nötig, dessen Berechtigungen übernommen werden.
\end{itemize}

\subsection*{Nachteile}
\begin{itemize}
	\item Das Übertragen von Berechtigungen von einem zum anderen Prozess wird tendenziell komplizierter/aufwändiger
\end{itemize}