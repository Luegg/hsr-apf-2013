\section{Role Rights Definition}
Beim Definieren von Sicherheitsrichtlinien spielt das \emph{Least Privilege} oder auch das \emph{Need to know} Prinzip eine fundamentale Rolle: Jedes Subjekt soll gerade so viele Berechtigungen erhalten, damit es seine Aufgaben ungehindert erledigen kann.

Das \emph{Role Rights Definition} Pattern beschreibt einen systematischen Ansatz, wie aus vorhandenen \emph{Requirements Engineering} Artefakten \emph{Need to Know}-konforme Sicherheitsregeln gewonnen werden können

\subsection*{Kontext}
Eine relativ komplexe Ansammlung von Rollen soll mit passenden Berechtigungen ausgestattet werden.

\subsection*{Problem}
\emph{Role Based Access Control} wird in vielen Systemen als grundlegendes Sicherheitkonzept verwendet. Wie im Abschnitt \ref{sec:rbac} erwähnt ist die Definition von Berechtigungskonzepten bei umfangreichen System (und grosser Anzahl an Aufgabenbereichen) mit beträchtlichem Aufwand verbunden.

Zudem überlässt \emph{Role Based Access Control} es komplett dem Implementator, aufgrund von welchen Informationen Gruppen resp. deren Berechtigungen zusammengestellt werden.


Wie können wir \emph{Role Based Access Control} mit Sicherheitsrichtlinien füttern, welche folgende Punkte befriedigen?
\begin{itemize}
	\item Rollen sollen Aufgabenbereichen in der Organisationsstruktur entsprechen
	\item Rechte sollen so erteilt werden, dass sie dem \emph{Need to know} Prinzip genügen
	\item Weiterhin soll die Anpassung bestehender Rollen und Rechten so einfach wie möglich bleiben
	\item Die Definition von Rechten und Rollen soll unabhängig von einer effektiven Implementierung des Systems bleiben
\end{itemize}


\subsection*{Lösung}

\begin{itemize}
	\item 
\end{itemize}

\subsection*{Erweiterungen}


\subsection*{Vor- \& Nachteile}
\begin{itemize}
	\item 
\end{itemize}

\subsection*{Beispielanwendungen}
\begin{itemize}
	\item 
\end{itemize}