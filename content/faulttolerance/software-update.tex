
\section{Software Update}

\subsection{Ausgangslage}

Trotz guter Qualitätssicherungsmethoden zur Vermeidung von Fehlverhalten kann es zu Fehlern kommen. Wodurch Software Updates nötig werden, diese dürfen jedoch das System nicht zu Anhalten zwingen und die damit verbundene Downtime minimieren.

\subsection{Lösungsansatz}

Vor der ersten Applikationsauslieferung muss geklärt sein wie sie erweitert bzw. erneuert werden kann, ohne das System zu stoppen.

Falls die neue Version einer Applikation zur gleichen Zeit wie die alte ausgeführt werden kann, ist es möglich einen Failover Routine laufen zu lassen die zwischen der alten und der neuen Version interagiert. Dies ermöglicht eine minimale Downtime. Zudem können automatisierte Akzeptanztests gestartet werden, welche prüfen ob die neue Version korrekt arbeitet. Falls dies nicht der Fall ist, kann mittels Recovery Blocks wieder zur alten Version zurück gewechselt werden.

In einem komplexen System können nicht alle Komponenten des Systems gleichzeitig ausgetauscht werden. Es kann also erforderlich sein, dass das neue Komponenten rückwärtskompatibel sein müssen. Dies kann unter anderem erreicht werden, indem ein Versionsindikator eingeführt wird und die neuen Funktionen Regeln enthalten, welche festlegen wie sie mit fehlenden oder geänderten Attributen umgehen soll.

\subsection{Schlussfolgerung}

Designen sie die Applikation so, das Änderungensmöglichkeiten bereits in ihrerem ersten Release miteingeflossen sind. Halten sie sich immer vor Augen wie sie die Software im ausgelieferten Zustand erweitern und verbessern können, da ein einbauen einer Update Routine nachdem Ausliefern keine einfache Sache ist.
