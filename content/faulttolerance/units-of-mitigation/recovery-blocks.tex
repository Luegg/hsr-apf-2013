\subsection{Recovery Blocks}


\subsubsection*{Einleitung}

Im Kapitel \url{http://wiki.ifs.hsr.ch/APF/U11_3_Redundancy} wurde bereits darauf hingewiesen, dass eine exakte Kopie auch die gleichen Fehler produzieren wird. Daher müssen alternative Lösungen für eine Problemstellung unterschiedlich implementiert werden. Dies kann wie folgt bewerkstelligt werden
\begin{itemize}
	\item Redundante Implementierungen parallel ausgeführt. Mittels Voting wir anschliessend das beste Resultat gewählt. (Nachteil: Overhead)
	\item Alternative werden nur ausgeführt, wenn das erste Resultat nicht befriedigend war, die ist die Recovery Block Strategie.
\end{itemize}

\subsubsection*{Beispiel}

\begin{verbatim}
  Try
    Heap sort
    if not okay then throw
  Catch
    Try
      Insertion sort
      if not okay then throw
    Catch
      Bubble sort
\end{verbatim}

\subsubsection*{Tradeoff}

\begin{itemize}
	\item Akzeptanz-Tests: Es ist nicht immer einfach zu entscheiden ob ein Resultat akzeptiert werden kann.
	\item Alternativen: Für gewisse Algorithmen gibt es keine oder nur unbefriedigende Alternativen.
	\item State: Der System State muss vor dem ersten Block gespeichert werden, damit für den nächsten Block die gleichen Bedingungen herrschen.
	\item Komplexität: Recovery Blocks an sich führen zu mehr Code und somit zu mehr Komplexität. Auch die vorherigen Punkte tragen zu einer höheren Komplexität bei.
\end{itemize}

\subsubsection*{Prüfungsfragen}

\begin{itemize}
	\item Um welchen Typ von Redundanz handelt es sich bei Recovery Blocks?
	\item Stellen Sie die Funktionsweise von Recovery Blocks anhand von Pseudocode und einem Fluss-Diagramm dar.
	\item Nennen Sie drei Nachteile von Recovery Blocks.
\end{itemize}

\begin{itemize}
	\item Nennen Sie drei Nachteile von Recovery Blocks.
\end{itemize}

