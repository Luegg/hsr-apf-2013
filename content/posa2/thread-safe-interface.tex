\section{Thread-Safe Interface}


The \emph{Thread-Safe Interface} design pattern minimizes locking overhead and ensures that intra-component method calls do not incur 'self-deadlock' by trying to reacquire a lock that is held by the component already.

\subsection{Kontext}

Komponenten in multi-threaded Applikationen, welche intra-Komponenten Methoden aufrufe benutzen.

\subsection{Problem}

Multi-Threaded Komponenten haben häufig mehrere öffentliche Interface Methoden (und private Implementations-Methoden), welche den Zustand der Komponente ändern können.
Um Race-Conditions zu verhindern, kann ein Lock intern für die Komponente verwendet werden, um Aufrufe solcher Methoden seriell zu machen.
Falls das gemacht wird, werden folgende Punkte bei intra-komponenten-methoden Aufrufen relevant:

\begin{itemize}
	\item Thread-Safe Komponenten müssen self-deadlock verhindern (wenn 2 Komponenten-Methoden aufgerufen werden und beide das gleiche Lock verwenden)
	\item Der overhead durchs Locking sollte so klein wie möglich gehalten werden
\end{itemize}

\subsection{Lösung}

\begin{itemize}
	\item Das Interface sollte das Lock machen, die eigentliche Implementation sollte erst nach dem Locken aufgerufen werden.
	\item Implementations-Methoden sollten nur etwas machen, wenn sie von den Interface-Methoden aufgerufen werden.
\end{itemize}

\subsection{Varianten}

\begin{itemize}
	\item Thread-Safe Facade: Kann verwendet werden, wenn der Zugriff auf ein ganzes Subsystem synchronisiert werden muss.
	\item Thread-Safe Wrapper Facade: Hilft den Zugriff auf eine nicht-synchronisierte Klasse zu synchronisieren
\end{itemize}

\subsection{Kown uses}

\begin{itemize}
	\item Java: java.util.Hashtable verwendet dies
	\item Security Checkpoints: Im Real-Life wird das z.B. beim Zoll verwendet: Man darf nur rein, wenn der Security Guard an der Grenze den einlass gewährt
\end{itemize}

\subsection{Vorteile}

\begin{itemize}
	\item Increased Robustness (kein self-deadlock)
	\item Verbesserte Performance (kein unnötiges acquiren/releasen von Locks)
	\item Vereinfachung der Software
\end{itemize}

\subsection{Nachteile}

\begin{itemize}
	\item Mehr Indirektionen und extra Methoden (Komplexität)
	\item Deadlock möglich (Thread-Safe Interface verhindert nicht einfach so die Deadlocks)
	\item Da Java/C++ auch class-level statt object-level access auf Methoden ermöglichen, kann es möglich sein, das falsch zu verwenden
	\item Overhead
\end{itemize}

\subsection{See also}

\begin{itemize}
	\item Decorator pattern
\end{itemize}

